% SID: 310182212
% Name: Tianyu Pu
\documentclass[12pt]{article}
\usepackage{amsmath}
\usepackage{amssymb}
\usepackage{amsthm}
\usepackage{fancyhdr}
\usepackage{graphicx}
\usepackage{natbib}
\usepackage{hyperref}
\usepackage{fullpage}
\pagestyle{fancy}
\rhead{Tianyu Pu 310182212}

\title{COMP3520 Assignment 2\\
  HOST Dispatcher Design Document}
\author{Tianyu Pu\\
  310182212}
\date{\today}
\fancyhead[R]{310182212}

\begin{document}
\maketitle

\section{Introduction}
In this report we shall examine the Hypothetical Operating System Testbed
(HOST) Dispatcher, starting wih a discussion of memory allocation
algorithms as well as data structures for the memory and resources, the overall
structure of the dispatcher, and concluding with a discussion and comparison
of the dispatching schemes of HOST and real operating systems.
\cite{ausmines}

\section{Memory allocation}
For the purposes of the assignment, the memory used by the processes is a
contiguous block of memory assigned to that process for the entire lifetime of
that process. The system only has 1024 megabytes (MB) of memory available for
processes, and 64 MB must always be available for real-time processes -- this
leaves 960 MB available for all active user jobs.

In addition, there is no virtual memory support, nor is the system paged.

Within these constraints, a number of variable memory partitioning algorithms
are possible. The most commonly used are described below.

\subsection{Overview of memory allocation algorithms}

\subsection{Justification of final choice}

\section{Data structures}

\subsection{Queuing}

\subsection{Dispatching}

\subsection{Allocating}

\section{Program structure}

\subsection{Overview}

\subsection{\tt{hostd.c}}

\subsection{\tt{pcb.c}}

\subsection{\tt{mab.c}}

\subsection{\tt{rsrc.c}}

\section{Dispatching schemes}

\subsection{Dispatching scheme of the scheduler}

\subsection{Schemes used by popular operating systems}

\subsection{Discussion}

\section{Concluding remarks}

\bibliographystyle{plainnat}
\bibliography{report}

\end{document}
